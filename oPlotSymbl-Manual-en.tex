\documentclass[
	a4paper,
	parskip=half,
    pagesize=auto,      		% schreibt die Papiergröße korrekt ins Ausgabedokument
    listof=totoc,   		% Verzeichnisse im Inhaltsverzeichnis
    bibliography=totoc,
	11pt
]{scrartcl}

\usepackage[T1]{fontenc}
\usepackage[utf8]{inputenc}
\usepackage[english]{babel}
\usepackage{babelbib}
\usepackage{graphicx}
\usepackage{tikz}
\usepackage{amsmath,amssymb,amsthm,textcomp}
\usepackage{enumerate}
\usepackage{multicol}
\usepackage{listings}
\usepackage{url}
\usepackage{setspace}
\usepackage{xcolor}
\usepackage{colortbl}
\usepackage{float}
\usepackage{oplotsymbl}

\usepackage[
pdftitle={oPlotSymbl Package Introduction}, 
pdfauthor={B. Michel Döhring, JLU Giessen},
colorlinks=true,linkcolor=blue,urlcolor=blue,citecolor=gray,bookmarks=true,
bookmarksopenlevel=2]{hyperref}

\usepackage{geometry}
\geometry{
	nomarginpar,
	left=30mm,right=20mm,top=20mm,bottom=20mm
}

%%%%%%%%%%%%%%%%%%%%%%%%%%%%%%%%%%%%%%%%%%%%%%
%%%% colours for listings %%%%
\definecolor{codegreen}{rgb}{0,0.6,0}
\definecolor{codegray}{rgb}{0.5,0.5,0.5}
\definecolor{codepurple}{rgb}{0.58,0,0.82}
\definecolor{backcolour}{rgb}{0.95,0.95,0.92}
%%%%%%%%%%%%%%%%%%%%%%%%%%%%%%%%%%%%%%%%%%%%%%
%%%%%%%%%%%%%%%%%%%%%%%%%%%%%%%%%%%%%%%%%%%%%%
%%%%%%%%%%%%%%%%%%%%%%%%%%%%%%%%%%%%%%%%%%%%%%
%%%% Style settings for listings %%%%
\lstdefinestyle{mystyle}{
language=[LaTeX],
    backgroundcolor=\color{backcolour},   
    commentstyle=\color{codegreen},
    keywordstyle=\color{magenta},
    numberstyle=\tiny\color{codegray},
    stringstyle=\color{codepurple},
    basicstyle=\footnotesize,
    breakatwhitespace=false,         
    breaklines=true,                 
    captionpos=b,                    
    keepspaces=true,                 
    numbers=left,                    
    numbersep=5pt,                  
    showspaces=false,                
    showstringspaces=false,
    showtabs=false,                  
    tabsize=2
}
%%%%%%%%%%%%%%%%%%%%%%%%%%%%%%%%%%%%%%%%%%%%%%
%%%%%%%%%%%%%%%%%%%%%%%%%%%%%%%%%%%%%%%%%%%%%%





%\linespread{1.3}

\newcommand{\linia}{\rule{\linewidth}{0.5pt} \rule{\linewidth}{0.5pt}}

% my own titles
\makeatletter
\renewcommand{\maketitle}{
\begin{center}
\vspace{2ex}
{\huge \textsc{\@title}}
\vspace{1ex}
\\
\linia\\
\huge{\@date}\\
\vspace{5ex}
\@author\\ 
E-Mail: \href{mailto:micheld.93@ gmail.com}{micheld.93@gmail.com}\\
\vspace{10ex}
\end{center}
}
\makeatother
%%%%%%%%%%%%%%%%%%%%%%%%%%%%%%%%%%%%%%%%%%

%%%----------%%%----------%%%----------%%%----------%%%

\begin{document}

\title{{\color[HTML]{FF0000} oPlotSymbl Package Introduction}}

\author{B. Michel Döhring}

\date{03/08/2017 (V1.3)}

\maketitle

%%%%%%%%%%%%%%%%%%%%%%%%%
%%% Table of Contents %%%
%%%%%%%%%%%%%%%%%%%%%%%%%
\begin{multicols}{2}
\tableofcontents
\end{multicols}
%%%%%%%%%%%%%%%%%%%%%%%%%
%%%%%%%%%%%%%%%%%%%%%%%%%

% 1 1/2 line space
\onehalfspacing 



% \clearpage

\section{Introduction}

This package is named "\textit{oPlotSymbl}" and it includes symbols, which are not easily available. Especially, these symbols are used in scientific plots, but the potential user is allowed to use in another way. The idea came to my mind during writing my bachelor thesis, where I needed many plots with many different symbols. 

This package can be loaded with the following command: 

\begin{lstlisting}
\usepackage{oplotsymbl}	
\end{lstlisting}

There are no additional options implemented yet. Now, it is important to me to mention the used packages. \textit{oPlotSymbl} uses \textit{TikZ} \cite{tikz} and so it loads the \textit{xcolor} package automatically. That means it is possible to use the whole beauty of \textit{xcolor}'s \cite{xcolor} colour palette.

\section{Version History}

I will collect all changes in this chapter, here. 

\subsection{Version 1.2 (28.01.2017)}

\begin{itemize}
    \item make the manuals's tex file available for everybody
    \item hope the final release for tex live is possible now
    \item some people ask to change the name to oPlotSymbol, but I don't see any advantages in it. Sorry.
    \item share the links on CTAN and GitHub
    \item some changes on the code itself but NO, absolutely NO changes for the user 
\end{itemize}


\subsection{Version 1.3 (03.08.2017)}

\begin{itemize}
    \item minor changes: manual
    \item bug fix for hexagofill
    \item some changes on the code itself but NO, absolutely NO changes for the user 
    \item I don't like version numbering like 1.2.3. Don't see any advantages in it for oplotsymbl
\end{itemize}


\section{Repository and Contact}

The repository/this package is available on GitHub and through CTAN \cite{ctan} and TeXLive \cite{texlive}. You will find it here:


\begin{itemize}
    \item \url{https://www.ctan.org/pkg/oplotsymbl}
    \item \url{https://github.com/micheld93/oPlotSymbl-LaTeX/}  
\end{itemize}


If you have suggestions, problems or you only want to say "Hi", then contact me at \href{mailto:micheld.93@ gmail.com}{micheld.93@gmail.com}.


\section{Symbols and Commands}

The following sub-sections include all defined symbols sorted in categories. The names are chosen to work with other packages which includes symbols. If you want to use these symbols in the running text, you will use two curved brackets directly after the command to have space between symbol and the following word. I tried to make this package as easy as possible to understand and use. This is why the commands are as close as possible to each other.  


\subsection{Triangle}

\begin{table}[H]
\centering
\begin{tabular}{|c||l|c|c||l|}
\hline
Symbol            & Command                       & Suffix & Explanation & Description                                       \\ \hline \hline
\trianglepa       & \lstinline!\trianglepa!     & pa     & peak above   & none                                              \\ \hline
\trianglepafill   & \lstinline!\trianglepafill! & pa     & peak above   & filled triangle                                   \\ \hline
\trianglepadot    & \lstinline!\trianglepadot!                & pa     & peak above   & triangle with dot                                 \\ \hline
\trianglepalinev  & \lstinline!\trianglepalinev!              & pa     & peak above   & triangle with vertical line                       \\ \hline
\trianglepalineh  & \lstinline!\trianglepalineh!              & pa     & peak above   & triangle with horizontal line                     \\ \hline
\trianglepalinevh & \lstinline!\trianglepalinevh!             & pa     & peak above   & triangle with both lines \\ \hline
\trianglepacross  & \lstinline!\trianglepacross!              & pa     & peak above   & triangle with cross                               \\ \hline
\trianglepafillha & \lstinline!\trianglepafillha!             & pa     & peak above   & half filled triangle (above)                      \\ \hline
\trianglepafillhb & \lstinline!\trianglepafillhb!             & pa     & peak above   & half filled triangle (below)                      \\ \hline
\trianglepafillhr & \lstinline!\trianglepafillhr!             & pa     & peak above   & half filled triangle (right)                      \\ \hline
\trianglepafillhl & \lstinline!\trianglepafillhl!             & pa     & peak above   & half filled triangle (left)                       \\ \hline
\end{tabular}
\end{table}



\subsubsection{Additional Triangles}

All other triangles follow the syntax shown above. It's always 

\begin{lstlisting}
	\triangle-suffixDESCRIPTION
\end{lstlisting}

"DESCRIPTION" is to exchange with terms like "cross" or "dot" etc. "-suffix" means the orientation of the triangle's highest peak. 
Other orientations are shown in the table below:


\begin{table}[H]
\centering
\begin{tabular}{|c|c|}
\hline
Suffix	&	Explanation	   \\ \hline \hline
pa & peak above \\ \hline
pb & peak below \\ \hline
pr & peak right \\ \hline
pl & peak left \\ \hline
\end{tabular}
\end{table}


\newpage
\subsection{Circle (here: Circlet)}

Some other packages use \lstinline{\circle} or \lstinline{\circ} for circles, so I decided to use \lstinline{\circlet} instead of other cryptic abbreviations.    

\begin{table}[H]
\centering
\begin{tabular}{|c||l||l|}
\hline
Symbol            & Command  &  Description            \\ \hline \hline
\circlet       & \lstinline!\circlet!        & none                                              \\ \hline
\circletfill   & \lstinline!\circletfill!     & filled circle(let)                                   \\ \hline
\circletdot    & \lstinline!\circletdot!                    & circle(let) with dot                                 \\ \hline
\circletlinev  & \lstinline!\circletlinev!                  & circle(let) with vertical line                       \\ \hline
\circletlineh  & \lstinline!\circletlineh!                  & circle(let) with horizontal line                     \\ \hline
\circletlinevh & \lstinline!\circletlinevh!                 & circle(let) with both lines \\ \hline
\circletcross  & \lstinline!\circletcross!                  & circle(let) with cross                               \\ \hline
\circletfillha & \lstinline!\circletfillha!                 & half filled circle(let) (above)                      \\ \hline
\circletfillhb & \lstinline!\circletfillhb!                 & half filled circle(let) (below)                      \\ \hline
\circletfillhr & \lstinline!\circletfillhr!                 & half filled circle(let) (right)                      \\ \hline
\circletfillhl & \lstinline!\circletfillhl!                 & half filled circle(let) (left)                       \\ \hline
\end{tabular}
\end{table}


\subsection{Pentagon (here: Pentago)}

The same problem as we know from circle/circlet happens with pentagon. I decided to use "pentago", so it's near enough to pentagon. 

\begin{table}[H]
\centering
\begin{tabular}{|c||l||l|}
\hline
Symbol            & Command  &  Description            \\ \hline \hline
\pentago       & \lstinline!\pentago!        & none                                              \\ \hline
\pentagofill   & \lstinline!\pentagofill!     & filled pentago                                   \\ \hline
\pentagodot    & \lstinline!\pentagodot!                    & pentago with dot                                 \\ \hline
\pentagolinev  & \lstinline!\pentagolinev!                  & pentago with vertical line                       \\ \hline
\pentagolineh  & \lstinline!\pentagolineh!                  & pentago with horizontal line                     \\ \hline
\pentagolinevh & \lstinline!\pentagolinevh!                 & pentago with both lines \\ \hline
\pentagocross  & \lstinline!\pentagocross!                  & pentago with cross                               \\ \hline
\pentagofillha & \lstinline!\pentagofillha!                 & half filled pentago (above)                      \\ \hline
\pentagofillhb & \lstinline!\pentagofillhb!                 & half filled pentago (below)                      \\ \hline
\pentagofillhr & \lstinline!\pentagofillhr!                 & half filled pentago (right)                      \\ \hline
\pentagofillhl & \lstinline!\pentagofillhl!                 & half filled pentago (left)                       \\ \hline
\end{tabular}
\end{table}




\subsection{Star (here: Starlet)}

\begin{table}[H]
\centering
\begin{tabular}{|c||l||l|}
\hline
Symbol            & Command  &  Description            \\ \hline \hline
\starlet       & \lstinline!\starlet!        & none                                              \\ \hline
\starletfill   & \lstinline!\starletfill!     & filled starlet                                   \\ \hline
\starletdot    & \lstinline!\starletdot!                    & starlet with dot                                 \\ \hline
\starletlinev  & \lstinline!\starletlinev!                  & starlet with vertical line                       \\ \hline
\starletlineh  & \lstinline!\starletlineh!                  & starlet with horizontal line                     \\ \hline
\starletlinevh & \lstinline!\starletlinevh!                 & starlet with both lines \\ \hline
\starletcross  & \lstinline!\starletcross!                  & starlet with cross                               \\ \hline
\starletfillha & \lstinline!\starletfillha!                 & half filled starlet (above)                      \\ \hline
\starletfillhb & \lstinline!\starletfillhb!                 & half filled starlet (below)                      \\ \hline
\starletfillhr & \lstinline!\starletfillhr!                 & half filled starlet (right)                      \\ \hline
\starletfillhl & \lstinline!\starletfillhl!                 & half filled starlet (left)                       \\ \hline
\end{tabular}
\end{table}


\subsection{Rhombus}

\begin{table}[H]
\centering
\begin{tabular}{|c||l||l|}
\hline
Symbol            & Command  &  Description            \\ \hline \hline
\rhombus       & \lstinline!\rhombus!        & none                                              \\ \hline
\rhombusfill   & \lstinline!\rhombusfill!     & filled rhombus                                   \\ \hline
\rhombusdot    & \lstinline!\rhombusdot!                    & rhombus with dot                                 \\ \hline
\rhombuslinev  & \lstinline!\rhombuslinev!                  & rhombus with vertical line                       \\ \hline
\rhombuslineh  & \lstinline!\rhombuslineh!                  & rhombus with horizontal line                     \\ \hline
\rhombuslinevh & \lstinline!\rhombuslinevh!                 & rhombus with both lines \\ \hline
\rhombuscross  & \lstinline!\rhombuscross!                  & rhombus with cross                               \\ \hline
\rhombusfillha & \lstinline!\rhombusfillha!                 & half filled rhombus (above)                      \\ \hline
\rhombusfillhb & \lstinline!\rhombusfillhb!                 & half filled rhombus (below)                      \\ \hline
\rhombusfillhr & \lstinline!\rhombusfillhr!                 & half filled rhombus (right)                      \\ \hline
\rhombusfillhl & \lstinline!\rhombusfillhl!                 & half filled rhombus (left)                       \\ \hline
\end{tabular}
\end{table}


\newpage
\subsection{Hexagon (here: Hexago)}

Well, we already know it. Hexagon is used in other packages, so there is a necessity to use different words. 

\begin{table}[H]
\centering
\begin{tabular}{|c||l||l|}
\hline
Symbol            & Command  &  Description            \\ \hline \hline
\hexago       & \lstinline!\hexago!        & none                                              \\ \hline
\hexagofill   & \lstinline!\hexagofill!     & filled hexago                                   \\ \hline
\hexagodot    & \lstinline!\hexagodot!                    & hexago with dot                                 \\ \hline
\hexagolinev  & \lstinline!\hexagolinev!                  & hexago with vertical line                       \\ \hline
\hexagolineh  & \lstinline!\hexagolineh!                  & hexago with horizontal line                     \\ \hline
\hexagolinevh & \lstinline!\hexagolinevh!                 & hexago with both lines \\ \hline
\hexagocross  & \lstinline!\hexagocross!                  & hexago with cross                               \\ \hline
\hexagofillha & \lstinline!\hexagofillha!                 & half filled hexago (above)                      \\ \hline
\hexagofillhb & \lstinline!\hexagofillhb!                 & half filled hexago (below)                      \\ \hline
\hexagofillhr & \lstinline!\hexagofillhr!                 & half filled hexago (right)                      \\ \hline
\hexagofillhl & \lstinline!\hexagofillhl!                 & half filled hexago (left)                       \\ \hline
\end{tabular}
\end{table}



\subsection{Square}

To avoid problems with other commands, I decided to use the frankenword \textbf{"squad"} (it's a composition of english \textit{square} and german or non-mathematical \textit{quadrat}).


\begin{table}[H]
\centering
\begin{tabular}{|c||l||l|}
\hline
Symbol            & Command  &  Description            \\ \hline \hline
\squad       & \lstinline!\squad!        & none                                              \\ \hline
\squadfill   & \lstinline!\squadfill!     & filled square                                   \\ \hline
\squaddot    & \lstinline!\squaddot!                    & square with dot                                 \\ \hline
\squadlinev  & \lstinline!\squadlinev!                  & square with vertical line                       \\ \hline
\squadlineh  & \lstinline!\squadlineh!                  & square with horizontal line                     \\ \hline
\squadlinevh & \lstinline!\squadlinevh!                 & square with both lines \\ \hline
\squadcross  & \lstinline!\squadcross!                  & square with cross                               \\ \hline
\squadfillha & \lstinline!\squadfillha!                 & half filled square (above)                      \\ \hline
\squadfillhb & \lstinline!\squadfillhb!                 & half filled square (below)                      \\ \hline
\squadfillhr & \lstinline!\squadfillhr!                 & half filled square (right)                      \\ \hline
\squadfillhl & \lstinline!\squadfillhl!                 & half filled square (left)                       \\ \hline
\end{tabular}
\end{table}



\subsection{Other Symbols}

\begin{table}[H]
\centering
\begin{tabular}{|c||l||l|}
\hline
Symbol            & Command  &  Description            \\ \hline \hline
\linev       & \lstinline!\linev!        & vertical line                                              \\ \hline
\lineh   & \lstinline!\lineh!     & horizontal line                                   \\ \hline
\scross    & \lstinline!\scross!                    & single cross                                 \\ \hline
\linevh  & \lstinline!\linevh!                  & vertical and horizontal line                       \\ \hline
\scrossvh  & \lstinline!\scrossvh!                  & single cross with lines                     \\ \hline
\end{tabular}
\end{table}


 
\section{Font Size}

All symbols use relative units for scaling. \LaTeX{} provides the unit "em" that means the width of the capital letter "M" in current font. \textit{oPlotSymbl} scales every symbol for you automatically and correctly. No need to worry. If you like to increase symbol size, then it's done with normal behavior for increasing font size. That's it.     

\section{Colours}

\textit{oPlotSymbl} uses the \textit{xcolor} package so it is possible to use all pre-defined colours from \textit{xcolor} \cite{xcolor}. 

You can colour the symbols very easily like this:

\begin{lstlisting}
\pentagofillhl[opurple]	
\end{lstlisting}

There, you get a purple half filled pentagon \pentagofillhl[opurple]. You can define own colours with the following command:

\begin{lstlisting}
\definecolor{colour's name}{colour palette}{specific code}
\end{lstlisting}

There, you can define your own name for a missing colour. I recommend to use RGB or HTML as "colour palette". Between the last brackets you have to put your specific code that is determined trough your picked  "colour palette". I will give an example to make the start with \textit{oPlotSymbl} as easy as possible. 

\begin{lstlisting}
	\definecolor{black}{HTML}{000000}
\end{lstlisting}

This listing gives us black. It uses a custom name, followed by the "colour palette" and then the colour code for chosen option. As shown above \textit{oPlotSymbl} follows normal \textit{xcolor} \cite{xcolor} commands.
  
In addition, some colours are pre-defined for my own needs. These colours are:

\begin{table}[H]
\centering
\begin{tabular}{|l|l|l|c|}
\hline
Colour                   & Colour Name   & Colour Name for Command & RGB Code \\ \hline \hline
\cellcolor[RGB]{0,0,0} & black         & oblack                  & 0,0,0    \\ \hline
\cellcolor[RGB]{255,0,0} & red           & ored                    & 255,0,0    \\ \hline
\cellcolor[RGB]{0,255,0} & green         & ogreen                  & 0,255,0    \\ \hline
\cellcolor[RGB]{0,0,255} & blue          & oblue                   & 0,0,255    \\ \hline
\cellcolor[RGB]{0,255,255} & cyan          & ocyan                   & 0,255,255    \\ \hline
\cellcolor[RGB]{255,0,255} & magenta       & omagenta                & 255,0,255    \\ \hline
\cellcolor[RGB]{255,255,0} & yellow        & oyellow                 & 255,255,0    \\ \hline
\cellcolor[RGB]{128,128,0} & dark yellow   & odyellow                & 128,128,0    \\ \hline
\cellcolor[RGB]{0,0,128} & mariner blue  & omblue                  & 0,0,128    \\ \hline
\cellcolor[RGB]{128,0,128} & purple        & opurple                 & 128,0,128    \\ \hline
\cellcolor[RGB]{128,0,0} & brown         & obrown                  & 128,0,0    \\ \hline
\cellcolor[RGB]{0,128,0} & olive green   & oolive                  & 0,128,0    \\ \hline
\cellcolor[RGB]{0,128,128} & dark cyan     & odcyan                  & 0,128,128    \\ \hline
\cellcolor[RGB]{0,0,160} & royel blue    & orblue                  & 0,0,160    \\ \hline
\cellcolor[RGB]{255,128,0} & orange        & oorange                 & 255,128,0   \\ \hline
\cellcolor[RGB]{128,0,255} & violet        & oviolet                 & 128,0,255    \\ \hline
\cellcolor[RGB]{255,0,128} & pink          & opink                   & 255,0,128    \\ \hline
\cellcolor[RGB]{255,255,255} & white         & owhite                  & 255,255,255    \\ \hline
\cellcolor[RGB]{192,192,192} & light grey    & olgrey                  & 192,192,192    \\ \hline
\cellcolor[RGB]{128,128,128} & grey          & ogrey                   & 128,128,128    \\ \hline
\cellcolor[RGB]{255,255,128} & light yellow  & olyellow                & 255,255,128    \\ \hline
\cellcolor[RGB]{128,255,255} & light cyan    & olcyan                  & 128,255,255    \\ \hline
\cellcolor[RGB]{255,128,255} & light magenta & olmagenta               & 255,128,255    \\ \hline
\cellcolor[RGB]{64,64,64} & dark grey     & odgrey                  & 64,64,64    \\ \hline
\end{tabular}
\end{table}



% #####################
% # Table of listings #
% #####################

% \lstlistoflistings

%%%%%%%%%%%%%%%%%%%%%%%%%%%%%%%%%%%%%%%%%%%%



% ##########################
% # Literature with BibTeX #
% ##########################


    \cleardoublepage
    \bibliography{literatur}
    \bibliographystyle{babunsrt-fl}

%%%%%%%%%%%%%%%%%%%%%%%%%%%%%%%%%%%%%%%%%%%%





\end{document}
